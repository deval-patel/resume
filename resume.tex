%%%%%%%%%%%%%%%%%%%%%%%%%%%%%%%%%%%%%%%
% Wenneker Resume/CV
% LaTeX Template
% Version 1.1 (19/6/2016)
%
% This template has been downloaded from:
% http://www.LaTeXTemplates.com
%
% Original author:
% Frits Wenneker (http://www.howtotex.com) with extensive modifications by 
% Vel (vel@LaTeXTemplates.com)
%
% License:
% CC BY-NC-SA 3.0 (http://creativecommons.org/licenses/by-nc-sa/3.0/
%
%%%%%%%%%%%%%%%%%%%%%%%%%%%%%%%%%%%%%%

%----------------------------------------------------------------------------------------
%	PACKAGES AND OTHER DOCUMENT CONFIGURATIONS
%----------------------------------------------------------------------------------------

\documentclass[a4paper,12pt]{memoir} % Font and paper size

%%%%%%%%%%%%%%%%%%%%%%%%%%%%%%%%%%%%%%%%%
% Wenneker Resume/CV
% Structure Specification File
% Version 1.1 (19/6/2016)
%
% This file has been downloaded from:
% http://www.LaTeXTemplates.com
%
% Original author:
% Frits Wenneker (http://www.howtotex.com) with extensive modifications by 
% Vel (vel@latextemplates.com)
%
% License:
% CC BY-NC-SA 3.0 (http://creativecommons.org/licenses/by-nc-sa/3.0/)
%
%%%%%%%%%%%%%%%%%%%%%%%%%%%%%%%%%%%%%%%%%

%----------------------------------------------------------------------------------------
%	PACKAGES AND OTHER DOCUMENT CONFIGURATIONS
%----------------------------------------------------------------------------------------

\usepackage{XCharter} % Use the Bitstream Charter font
\usepackage[utf8]{inputenc} % Required for inputting international characters
\usepackage[T1]{fontenc} % Output font encoding for international characters
\usepackage[top=0.8cm,left=1cm,right=1cm,bottom=0.8cm]{geometry} % Modify margins

\usepackage{graphicx} % Required for figures
\usepackage{fontawesome} % Required for Logos

\usepackage{hyperref} % URLs

\usepackage[usenames,dvipsnames]{xcolor} % Required for custom colours

\usepackage{tikz} % Required for the horizontal rule
\usepackage{enumitem} % Required for modifying lists
\usepackage{multicol}
\setlist{noitemsep,nolistsep} % Remove spacing within and around lists


\pagestyle{empty} % Disable all page numbering

\setlength{\parindent}{0pt} % Stop paragraph indentation

%----------------------------------------------------------------------------------------
%	NEW COMMANDS
%----------------------------------------------------------------------------------------

\newcommand{\cvheading}[2]{{\centering\Huge\bfseries\color{RoyalBlue} #1 \color{Gray}| #2\par}} % New command for the CV heading
\newcommand{\cvsubheading}[1]{{\centering\footnotesize\bfseries #1 \par}} % New command for the CV subheading
\newcommand{\BigSep}{\vspace{8em}} % New command for the spacing between headings
\newcommand{\Sep}{\vspace{0.15em}} % New command for the spacing between headings
\newcommand{\SmallSep}{\vspace{0.27em}} % New command for the spacing within headings

\newcommand{\CVSection}[1]{ % New command for the headings within sections
	{\large\textbf{\color{RoyalBlue}#1}}\par
	\SmallSep% Used for spacing
}

\newcommand{\CVItem}[3][]{ % New command for the item descriptions
	\textbf{#2}\par
	#1
	#3
	\SmallSep% Used for spacing
}
\newcommand{\CVSkill}[2]{ % New command for the item descriptions
	\textbf{#1}: #2\par
	\SmallSep% Used for spacing
}
\newcommand{\bluebullet}{\tikz\draw[RoyalBlue,fill=RoyalBlue] (0,0) circle (0.4ex);} % New command for the blue bullets

 % Include the file specifying document layout and packages

% %----------------------------------------------------------------------------------------
\begin{document}

%----------------------------------------------------------------------------------------
%	HEADING
%----------------------------------------------------------------------------------------

\cvheading{Deval Patel}{Software Engineer}% Large heading - your name
\Sep
\cvsubheading{\faEnvelope\hspace{0.5em}devalrocket@gmail.com\hspace{0.5em}\bluebullet\hspace{0.5em}\faPhone\hspace{0.5em}(647)-470-7505 \hspace{0.5em}\bluebullet\hspace{0.5em}\faGithub \hspace{0.5em}\href{https://github.com/deval-patel}{deval-patel}\hspace{0.5em}\bluebullet\hspace{0.5em}\faLinkedin\hspace{0.2em}\href{https://www.linkedin.com/in/deval-patel-82609b21a/}{Deval Patel}}
\SmallSep

%----------------------------------------------------------------------------------------
%	EXPERIENCE
%----------------------------------------------------------------------------------------

\CVSection{Experience}

%------------------------------------------------

\CVItem[\textit{Embedded ML Software Engineer}\hfill May 2022 - Present\par]{Qualcomm}{
	\begin{itemize}
		\item Led eNPU (embedded neural processing unit) driver design in \textbf{C} for Qualcomm's LPAI (low power AI) automotive chip through HPG-based modifications for multi-master DSP support, enabling faster iteration development and 3 months ahead of pre-silicon schedule.
		\item Achieved cross-platform support (mobile, automotive, IoT, XR/VR) for the eNPU driver through code refactoring and streamlining, resulting in a simpler, unified codebase.
		\item Optimized eNPU driver code for ultra-low latency (2-3 microseconds) on an \textbf{RTOS}, achieving industry-leading performance and real-time processing.
		\item Reduced eNPU driver TCM memory footprint by 50\% through LPI/non-LPI code separation.
		\item Performed pre-silicon verification on FPGAs identifying and rectifying driver issues, accelerating SoC bring-up and reducing development time.
		\item Built a system for high fidelity eAI model profiling (hardware vs. software scheduling) to identify latency bottlenecks in customer models. Automated this workflow using a custom \textbf{Python} script invoking \textbf{ADB} shell commands (model/binary deployment, test execution, CSV data collection, Excel report generation).
		\item Delivered a well-received presentation to internal and external teams on next-gen automotive chip driver development, enhancing their understanding of new architecture and functionalities.
	\end{itemize}
}

%------------------------------------------------

%------------------------------------------------

\CVItem[\textit{Teaching Assistant for Operating Systems/Software Design/Intro. to CS}\hfill Jan 2020 - May 2022\par]{University of Toronto}{
	% Courses Taught: 1 x Operating Systems (CSC369), 3 x Software Design (CSC207), 1 x Intro. to Computer Science (CSC148),\\1 x Intro. to Computer Programming (CSC108)
	\begin{itemize}
		\item Led 2 hour practical sessions communicating troubling concepts and guiding 30 students through exercises while encouraging a positive and collaborative environment.
		\item Solely created major assignments with \textbf{Python} and \textbf{Java}.
		\item Created automarking test suites using \textbf{PyTest}, \textbf{JUnit} and \textbf{Bash} resulting in a 200\% increase in\\marking efficiency.
	\end{itemize}
}

%------------------------------------------------

\Sep % Extra whitespace after the end of a major section

%----------------------------------------------------------------------------------------
%	Projects
%----------------------------------------------------------------------------------------

\CVSection{Projects}


%--------------------------------------------------------------

\CVItem[\textit{Lead Developer}\hfill Jan 2021 - May 2022\par]{\href{https://utapcsc.utm.utoronto.ca/utap/}{TA Application System}}{
	\begin{itemize}
		% \item Enforced accessibility compliance on a codebase used for the TA Application System.
		\item Migrated an existing \textbf{HTML} application to \textbf{React} and created reusable components resulting in a reduction of duplicate code.
		\item Added REST API endpoints allowing contracts to be generated, downloaded, signed and uploaded using \textbf{ExpressJS} and \textbf{PostgreSQL} resulting in a 200\% reduction of manual labour.
		\item Developed \textbf{SQL} queries enabling instructors to parse student applications by grade, time availability, and many other filters.

	\end{itemize}
}
% ----------------------------------

\CVItem[\textit{Co-Creator/Developer} \hfill Jan 2021 - May 2021\par]{Carden}
{
	\begin{itemize}
		\item Created an appealing website using \textbf{React} and \textbf{MaterialUI} allowing users to create and send interactive e-cards at the time of a global pandemic.
		\item Stored and retrieved user data, media and e-cards using \textbf{MongoDB} and \textbf{AWS S3}.
		%       % \item Stored images and videos using \textbf{AWS S3} allowing users to embed media into the e-cards.
		% \item Allowed users to embed media into e-cards by storing uploaded images and videos into \textbf{AWS S3}.
		% \item Containerized the application with \textbf{Docker} enabling a consistent development and deployment environment using \textbf{AWS Elastic Beanstalk}.
	\end{itemize}
}

`%------------------------------------------------

\Sep % Extra whitespace after the end of a major section

%----------------------------------------------------------------------------------------
%	EDUCATION
%----------------------------------------------------------------------------------------

\CVSection{Education}

%------------------------------------------------

\CVItem[HBSc. Computer Science Specialist - GPA: 3.52 \hfill September 2018 - April 2022 \par]{University of Toronto}

%------------------------------------------------

\Sep % Extra whitespace after the end of a major section

%----------------------------------------------------------------------------------------
%	SKILLS
%----------------------------------------------------------------------------------------

\CVSection{Skills and Technologies}
\vspace{-3mm}
% %------------------------------------------------
\begin{multicols}{2}
	\CVSkill{Software Tools}{Docker, AWS, Make/Cmake, gdb}
	\CVSkill{Front-end}{HTML, CSS, React, Express, Vue}
	\columnbreak
	\CVSkill{Programming}{C/C++, Python, JavaScript, Java}
	\CVSkill{Back-end}{REST, SQL, MongoDB, Neo4J}
\end{multicols}
%------------------------------------------------
% %------------------------------------------------

\end{document}
